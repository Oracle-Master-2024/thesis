
OpenJDK~\cite{openjdk} is a set of tools for creating and running Java programs, maintained by Oracle. HotSpot~\cite{hotspot} is one of these tools and the reference implementation of the Java Virtual Machine (JVM)~\cite{JVM}. Specifically, a virtual machine (VM) emulates a distinct computer in order to run various programs, which could be anything from full operating systems - to more specialized machines. The JVM is designed specifically to execute Java programs. It translates programs to instructions for the underlying machine, creating an abstraction of the hardware of the physical machine and allows Java programs to be run on any platform that the JVM runs on.

HotSpot is compromised of several components, such as an interpreter, a Just-In-Time (JIT) compiler, and a garbage collector (GC). In combination, these components provide the means for running different types of Java programs on the platforms supported by HotSpot.

HotSpot provides several garbage collectors, each with different characteristics and performance profiles. Different garbage collectors are optimized for different use cases, such as throughput and latency, and can be tuned to varying degrees. One of the garbage collectors in HotSpot is the Z Garbage Collector (ZGC)~\cite{zgc}. It was introduced as an experimental feature in OpenJDK 11, and declared production ready in OpenJDK 15. ZGC is modern, generational, region-based, concurrent garbage collector that aims to keep pause times constant at any given heap size.

%%% Local Variables:
%%% mode: latex
%%% TeX-master: "main"
%%% End:

