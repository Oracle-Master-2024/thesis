\label{background:zpages}
ZGC is a region-based garbage collector, which means it allocates objects inside regions, referred to as pages in ZGC. When new objects are allocated, they are allocated inside pages with respect to their size. ZGC categorizes pages into three types: \textit{Small}, \textit{Medium} or \textit{Large}, which come in different sizes and allow allocating objects in specific size ranges, as illustrated in Table~\ref{table:zpage_sizes}. While the smallest allocation size is 16 bytes, it is still possible for mutator threads to request smaller amounts of memory. However, all alocations will be aligned to a multiple of 16 bytes since the accuracy of 64-bit pointers will not allow for smaller allocation.

\begin{table}[H]
    \centering
    \begin{tabular}{lllll}
        \hline
        Page Type & Page Size          & \multicolumn{3}{l}{Object Size}        \\ \hline
        Small     & 2 MB                & \multicolumn{3}{l}{{[}16B, 256KB{]}}   \\
        Medium    & 32 MB               & \multicolumn{3}{l}{(256KB, 4MB)}       \\
        Large     & $\geq$ 4 MB & \multicolumn{3}{l}{$\geq$ 4MB} \\ \hline
    \end{tabular}
    \caption{Page sizes in ZGC. (Figure taken from~\cite{zpage_size_table}). }
    \label{table:zpage_sizes}
\end{table}

An example of what a Z-Page could look like, as well as the overhead data that is available for the garbage collector, is shown in Figure~\ref{fig:zpages}. There are some key aspects of the page that allow the garbage collector to efficiently do allocations, as well as work concurrently. The following important class members exist, and are labeled in the figure:

\begin{description}
    \item[Bump Pointer]
        The bump pointer is a pointer that exists within the page's memory, and lets the garbage collector know where the next allocation will happen. The bump pointer is moved forward every time an object is allocated in the page, and will therefore guarantee that any space from the bump pointer to the end of the page's memory will be available for allocation. In Figure~\ref*{fig:zpages}, the bump pointer is displayed on the bottom of the four allocated objects, meaning any future allocations will be done from that point forward.
    \item[Live Map]
        The live map is used by the garbage collector to keep track of which allocations are currently live. On the right hand side of Figure~\ref*{fig:zpages}, the livemap is shown. As the garbage collector is a mark-and-compact collector, the live map is not used to identify which objects to free from memory, but instead identify which objects are live, and should be relocated to other pages when memory needs to be freed. In the example, there are three live objects, and one dead. The liveness of each object is stored in the live map as a one or a zero at the start of the allocation.
    \item[Sequence Number]
        Every page has a sequence number which denotes during which garbage collection cycle a page was created. New allocations are only done in pages that have been created after the latest garbage collection cycle. Pages which are created before the garbage collection cycle are therefore separated from the new ones, allowing the garbage collector to work concurrently while mutator threads can still allocate new objects in separate pages. For example, if the current Java program has performed a garbage collection 5 times, any allocations done after that will only be done on pages with the sequence number 5. And if the program decides to run a 6th garbage collection, only garbage will be collected from pages with sequence number 0-5, and concurrent mutator threads will allocate new objects to pages with sequence number 6.
\end{description}

\begin{figure}[H]
    \centering
    \includesvg[scale=0.8]{figures/zpage_members3}
    \caption[]
    {An example of a Z-Page and the overhead data that is used by ZGC. The live map is simplified to only contain a single bit per object for strong references.} 
    \label{fig:zpages}
\end{figure}

\subsubsection{The GC Cycle}
\label{background:relocation}
A timeline of all allocations made on the heap inside Z-Pages is shown in Figure~\ref{fig:zgc_timeline}. The timeline shows the different phases of a garbage collection cycle, and how the garbage collector prepares for relocating memory in order to free unused memory.

\begin{figure}[H]
    \centering
    \begin{subfigure}[t]{.214\textwidth}
        \centering
        \includesvg[width=1\textwidth]{figures/zrel1}
        \caption{Initial state of the heap before the GC cycle}
        \label{fig:zrel1}
    \end{subfigure}%
    \hfill\vline\hfill
    \begin{subfigure}[t]{.32\textwidth}
        \centering
        \includesvg[width=1\textwidth]{figures/zrel2}
        \caption{The state of the heap after ZGC has started marking.}
        \label{fig:zrel2}
    \end{subfigure}%
    \hfill\vline\hfill
    \begin{subfigure}[t]{.32\textwidth}
        \centering
        \includesvg[width=1\textwidth]{figures/zrel3}
        \caption{The state of the heap after ZGC has finished marking the live objects}
        \label{fig:zrel3}
    \end{subfigure}%
    \caption{An overview of the heap for different phases in a GC cycle.}
    \label{fig:zgc_timeline}
\end{figure}

\begin{enumerate}
    \item In Figure~\ref*{fig:zrel1}, the initial state of the heap before a garbage collection is shown. One page has about 30\% free memory, while the other has about 50\%. The current GC cycle is 1, as shown by the top left digit of each page.
    \item In Figure~\ref*{fig:zrel2}, the GC cycle has started, and the old pages have been frozen. This means that any concurrent allocations will be done in new pages. This is represented by the blue frozen pages, and the new gray page with a new sequence number.
    \item In Figure~\ref*{fig:zrel3}, the GC has finished the liveness marking of the objects located in the first two pages. This means that there is now data that tells us where objects are allocated inside the pages, as well as their size.
\end{enumerate}

The way ZGC frees the memory that has been marked as garbage is by copying the live objects from one page to another which leads to the previously fragmented memory to be compacted. This process is referred to as relocation. When performing relocation, there are three different scenarios that can occur. The first scenario, where a successful relocation is done by copying items to a new page, is illustrated in Figure~\ref{fig:zrel_new}. This type of relocation requires the heap to have enough free space available to create a new page during the time of relocation. In Figure~\ref*{fig:zrel_new1} the new page is created, and in Figure~\ref*{fig:zrel_new3} the old page is freed from memory.

\begin{figure}[H]
    \centering
    \begin{subfigure}[t]{.2\textwidth}
        \centering
        \includesvg[width=1\textwidth,height=1.2885\textwidth]{figures/zrel_new1}
        \caption{A page has been selected for relocation.}
        \label{fig:zrel_new1}
    \end{subfigure}%
    \hfill\vline\hfill
    \begin{subfigure}[t]{.2\textwidth}
        \centering
        \includesvg[width=1\textwidth]{figures/zrel_new2}
        \caption{ZGC allocates a new page with a new sequence number that will be used as a relocation target.}
        \label{fig:zrel_new2}
    \end{subfigure}%
    \hfill\vline\hfill
    \begin{subfigure}[t]{.2\textwidth}
        \centering
        \includesvg[width=1\textwidth]{figures/zrel_new3}
        \caption{The objects from the first page are relocated to the new page.}
        \label{fig:zrel_new3}
    \end{subfigure}%
    \hfill\vline\hfill
    \begin{subfigure}[t]{.2\textwidth}
        \centering
        \includesvg[width=1\textwidth]{figures/zrel_new4}
        \caption{Once all objects have been copied over to the new page, the old page can be freed from memory.}
        \label{fig:zrel_new3}
    \end{subfigure}%
    \caption{An example of how a successful relocation is done when the heap has enough space to allocate a new page.}
    \label{fig:zrel_new}
\end{figure}

In the previous example, the relocation was successful in allocating a new page. If the heap instead does not have enough space to allocate a new page, the program will instead resort to doing an in place compaction. An in place compaction is a more expensive operation, and requires the thread performing it to write to the page's memory, meaning any other page trying to read from it must be paused. Consequently, this removes any concurrent execution inside of the page. In Figure~\ref{fig:zrel_in}, the process of in place compacting a page is illustrated. 

\begin{figure}[H]
    \centering
    \begin{subfigure}[t]{.2\textwidth}
        \centering
        \includesvg[width=1\textwidth,height=1.2885\textwidth]{figures/zrel_in1}
        \caption{The first live object in the page is being moved to the top of the page.}
        \label{fig:zrel_in1}
    \end{subfigure}%
    \hfill\vline\hfill
    \begin{subfigure}[t]{.2\textwidth}
        \centering
        \includesvg[width=1\textwidth,height=1.2885\textwidth]{figures/zrel_in2}
        \caption{The second live objects is moved as far upward as possible}
        \label{fig:zrel_in1}
    \end{subfigure}%
    \hfill\vline\hfill
    \begin{subfigure}[t]{.2\textwidth}
        \centering
        \includesvg[width=1\textwidth,height=1.2885\textwidth]{figures/zrel_in3}
        \caption{All objects are moved, and any fragmentation is moved below all live objects.}
        \label{fig:zrel_in1}
    \end{subfigure}%
    \hfill\vline\hfill
    \begin{subfigure}[t]{.2\textwidth}
        \centering
        \includesvg[width=1\textwidth,height=1.2885\textwidth]{figures/zrel_in4}
        \caption{The page is reset as a new page, with a new bump pointer that more compactly stores the live objects.}
        \label{fig:zrel_in1}
    \end{subfigure}%
    \caption{An example of how in place compaction is done to reduce fragmented memory.}
    \label{fig:zrel_in}
\end{figure}
 
%%% Local Variables:
%%% mode: latex
%%% TeX-master: "main"
%%% End: