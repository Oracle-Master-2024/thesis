ZGC is a region-based garbage collector, which means it allocates objects inside units of memory, referred to as Z-pages in ZGC. When new objects are allocated, they are allocated inside Z-pages with respect to their size. ZGC categorizes Z-pages into three types: \textit{Small}, \textit{Medium} or \textit{Large}, which come in different sizes and allow allocating objects in specific size ranges, as illustrated in Table~\ref{table:zpage_sizes}. While the smallest allocation size is 16 bytes, it is still possible for mutator threads to request smaller amounts of memory. However, all allocations will be aligned up to at least 16 bytes, and if higher, the aligned size will be rounded up to the nearest multiple of 8. The starting address of all allocations will also be a multiple of 8 bytes offset from the start of the Z-page's memory.

\begin{table}[H]
    \centering
    \begin{tabular}{lllll}
        \hline
        Z-Page Type & Z-Page Size   & \multicolumn{3}{l}{Object Size}      \\ \hline
        Small       & 2 MB          & \multicolumn{3}{l}{{[}16B, 256KB{]}} \\
        Medium      & 32 MB         & \multicolumn{3}{l}{(256KB, 4MB)}     \\
        Large       & $\geq$ 4 MB   & \multicolumn{3}{l}{$\geq$ 4MB}       \\ \hline
    \end{tabular}
    \caption{Z-Page sizes in ZGC. (Figure taken from~\cite{zpage_size_table}). }
    \label{table:zpage_sizes}
\end{table}

An example of what a Z-page could look like, as well as the overhead data that is available for the garbage collector, is shown in Figure~\ref{fig:zpages}. There are some key aspects of Z-pages that allow the garbage collector to efficiently do allocations, as well as work concurrently. Some of the more important attributes of the Z-pages are the, \textit{Bump Pointer}, \textit{Live Map}, \textit{Sequence Number}, and \textit{Age}.

\begin{description}
    \item[Bump Pointer]
        The bump pointer is a pointer that exists within the page's memory, and lets the garbage collector know where the next allocation will happen. The bump pointer is explained in greater detail in Section~\ref{sec:bump_pointer}. All allocations are done using bump pointers while using ZGC.
    \item[Live Map]
        The live map is used by the garbage collector to keep track of which allocations are currently live. Live objects are objects that have been marked as reachable by the program, which means that any objects that are not marked as reachable are considered dead. On the right hand side of Figure~\ref*{fig:zpages}, the live map is shown. As ZGC is a mark-and-compact garbage collector, it consists of a marking phase and a compacting phase. The live map is constructed during the marking phase of the garbage collection cycle. The live map is then used during the compacting phase, since that is when the live map will have updated knowledge about which objects are still reachable. The live map is stored in a bit array, where each bit represents an 8 Byte accuracy pointer into the page. If the bit is set to 1, there is a live object at that position. The live map is used to quickly find live objects during the compacting phase.
    \item[Sequence Number]
        Every Z-page has a sequence number which denotes during which garbage collection cycle it was created. New allocations are only done in Z-pages that have been created after the latest garbage collection cycle. Z-pages which are created before the garbage collection cycle are therefore separated from the new ones, allowing the garbage collector to work concurrently while mutator threads are still running. For example, if the current Java program has performed a garbage collection 5 times, any allocations done after that will only be done on pages with the sequence number 5. And if the program decides to run a 6th garbage collection, only garbage will be collected from pages with sequence number 0-5, and concurrent mutator threads will allocate new objects to pages with sequence number 6.
    \item[Age]
        ZGC is a generational garbage collector, meaning it has knowledge about how long some objects have been alive. This is done to optimize the garbage collection process, since object with a short lifespan tend to behave differently than objects with a long lifespan. ZGC does this on a page-by-page basis, where every object which is allocated in a page of memory have the same age, and so decides the age of all objects using one age counter per page. There are 3 different types of ages that an object can have, \textit{eden}, \textit{survivor}, and \textit{old}. The eden age is the age of objects that have been allocated in a page for the first time. The survivor age is the age of objects that have survived one or more garbage collection cycles without becoming unreachable. The old age is the age of objects that have surpassed a threshold for being a survivor and instead have been promoted to old. This functionality allows the garbage collector to not perform collections on old pages as often as on eden pages, since objects in eden pages tend to build up wasted space quicker.
\end{description}

\begin{figure}[H]
    \centering
    \includesvg[scale=0.8]{figures/zpage_withage}
    \caption[]
    {An example of a Z-page and the overhead data that is used by ZGC. The live map is simplified to only contain a single bit per object for marking liveness. There are three live objects in the page.} 
    \label{fig:zpages}
\end{figure}

\subsubsection{The Garbage Collection Cycle}
To explain the garbage collection cycle, it is best done with an example. A timeline of the garbage cycle and an example heap made up of Z-pages is shown in Figure~\ref{fig:zgc_timeline}. The timeline shows the different phases of a garbage collection cycle, and how the garbage collector prepares for relocating memory in order to free unused memory.

\begin{figure}[H]
    \centering
    \begin{subfigure}[t]{.214\textwidth}
        \centering
        \includesvg[width=1\textwidth]{figures/zrel1}
        \caption{Initial state of the heap before the GC cycle}
        \label{fig:zrel1}
    \end{subfigure}%
    \hfill\vline\hfill
    \begin{subfigure}[t]{.32\textwidth}
        \centering
        \includesvg[width=1\textwidth]{figures/zrel2}
        \caption{The state of the heap after ZGC has started marking.}
        \label{fig:zrel2}
    \end{subfigure}%
    \hfill\vline\hfill
    \begin{subfigure}[t]{.32\textwidth}
        \centering
        \includesvg[width=1\textwidth]{figures/zrel3}
        \caption{The state of the heap after ZGC has finished marking the live objects}
        \label{fig:zrel3}
    \end{subfigure}%
    \caption{An overview of the heap for different phases in a GC cycle.}
    \label{fig:zgc_timeline}
\end{figure}

\begin{enumerate}
    \item In Figure~\ref*{fig:zrel1}, the initial state of the heap before the first garbage collection is shown. One page has about 30\% free memory, while the other has about 50\%. The current GC cycle is 1, as shown by the top left digit of each page.
    \item In Figure~\ref*{fig:zrel2}, the GC cycle has started, and the old pages have been frozen. This means that any concurrent allocations will be done in new pages. This is represented by the blue frozen pages, and the new gray page with a new sequence number.
    \item In Figure~\ref*{fig:zrel3}, the GC has finished the marking phase of the garbage collection cycle. This means that there is now liveness data that lets ZGC know where objects are allocated inside the Z-pages.
\end{enumerate}

The way ZGC frees the memory that has been marked as garbage is by copying the live objects from one page to another which leads to the previously fragmented memory to be compacted, which for example makes it possible to fit the objects of two different pages into a single one. This process is referred to as relocation. When performing relocation, there are two different scenarios that can occur. The first scenario, where a successful relocation is done by copying items to a new page, is illustrated in Figure~\ref{fig:zrel_new}. This type of relocation requires the heap to have enough free space available to create a new page during the time of relocation. In Figure~\ref*{fig:zrel_new1} a new page is created, and in Figure~\ref*{fig:zrel_new3} the old page is freed from memory, meaning the total amount of memory is not decreased, but the memory was compacted.

\begin{figure}[H]
    \centering
    \begin{subfigure}[t]{.2\textwidth}
        \centering
        \includesvg[width=1\textwidth,height=1.2885\textwidth]{figures/zrel_new1}
        \caption{A page has been selected for relocation.}
        \label{fig:zrel_new1}
    \end{subfigure}%
    \hfill\vline\hfill
    \begin{subfigure}[t]{.2\textwidth}
        \centering
        \includesvg[width=1\textwidth]{figures/zrel_new2}
        \caption{ZGC allocates a new page with a new sequence number that will be used as a relocation target.}
        \label{fig:zrel_new2}
    \end{subfigure}%
    \hfill\vline\hfill
    \begin{subfigure}[t]{.2\textwidth}
        \centering
        \includesvg[width=1\textwidth]{figures/zrel_new3}
        \caption{The objects from the first page are relocated to the new page.}
        \label{fig:zrel_new3}
    \end{subfigure}%
    \hfill\vline\hfill
    \begin{subfigure}[t]{.2\textwidth}
        \centering
        \includesvg[width=1\textwidth]{figures/zrel_new4}
        \caption{Once all objects have been copied over to the new page, the old page can be freed from memory.}
        \label{fig:zrel_new3}
    \end{subfigure}%
    \caption{An example of how a successful relocation is done when the heap has enough space to allocate a new page.}
    \label{fig:zrel_new}
\end{figure}

In the previous example, the relocation was successful in allocating a new page to relocate objects into. If the heap instead does not have enough space to allocate a new page, the program will resort to performing an in place compaction. An in place compaction is an expensive operation that moves the objects inside of a page without a separate page as a destination. It requires the thread performing it to write to the page's own memory, meaning any other threads trying to read from the page must be paused. Consequently, this removes any concurrent execution inside of the page. In Figure~\ref{fig:zrel_in}, the process of in place compacting a page is illustrated. 

\begin{figure}[H]
    \centering
    \begin{subfigure}[t]{.2\textwidth}
        \centering
        \includesvg[width=1\textwidth,height=1.2885\textwidth]{figures/zrel_in1}
        \caption{The first live object in the page is being moved to the top of the page.}
        \label{fig:zrel_in1}
    \end{subfigure}%
    \hfill\vline\hfill
    \begin{subfigure}[t]{.2\textwidth}
        \centering
        \includesvg[width=1\textwidth,height=1.2885\textwidth]{figures/zrel_in2}
        \caption{The second live objects is moved as far upward as possible}
        \label{fig:zrel_in1}
    \end{subfigure}%
    \hfill\vline\hfill
    \begin{subfigure}[t]{.2\textwidth}
        \centering
        \includesvg[width=1\textwidth,height=1.2885\textwidth]{figures/zrel_in3}
        \caption{All objects are moved, and any fragmentation is moved below all live objects.}
        \label{fig:zrel_in1}
    \end{subfigure}%
    \hfill\vline\hfill
    \begin{subfigure}[t]{.2\textwidth}
        \centering
        \includesvg[width=1\textwidth,height=1.2885\textwidth]{figures/zrel_in4}
        \caption{The page is reset as a new page, with a new bump pointer that more compactly stores the live objects.}
        \label{fig:zrel_in1}
    \end{subfigure}%
    \caption{An example of how in place compaction is done to reduce fragmented memory.}
    \label{fig:zrel_in}
\end{figure}
 
%%% Local Variables:
%%% mode: latex
%%% TeX-master: "main"
%%% End:
