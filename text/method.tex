
%%% Local Variables:
%%% mode: latex
%%% TeX-master: "main"
%%% End:

\subsection{Exploratory Programming}
The main method for implementing new changes to the JVM code was using exploratory programming. This method implies doing changes to the code and seeing what impacts the changes have based on data that is collected from the program. Some tools used to facilitate this process were debuggers, and logging tools. The usage of these tools are explained in more detail below.
\subsubsection{Debuggers}
A debugger is a tool that is able to run your program with an interface to let the developer know the values of certain variables, or locations of memory addresses. A useful tool used during this project is the \textit{rr\_debugger}, designed by Robert O'Callahan et al.~\cite{rrdebugger}, which can be used to record and replay a program's execution with deterministic behavior. This makes it possible to view parts of the program at different times. This is very useful in the context of the Open JDK, since the code consists of several tens of thousands lines of code, which makes it easy to have different parts of the system interacting with each other without the developer knowing. Finding which areas in the code is causing the problem is very important for successfully debugging a faulty program, and this is what the debugger is used for.

\subsubsection{Logging}
The Open JDK has access to some useful logging tools, which simplifies the act of working with a very large code base. The JVM has been implemented with various points in the code where the program outputs information about the program's performance and state. This can be toggled on or off with the use of compiler flags in Java. Logging will be used in order to gather information about the program, and help understand what changes to the code might impact the behavior of the program.
