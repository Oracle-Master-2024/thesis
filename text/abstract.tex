
%%% Local Variables:
%%% mode: latex
%%% TeX-master: "main"
%%% End:
Memory fragmentation is an issue in usually common in programs that run for longer periods of time. By continuously alternating between allocating and releasing memory, the program can cause the memory to become fragmented, meaning objects are no longer stored in a contiguous block of memory. The free regions of memory between objects can then be hard to utilize, because certain objects might not fit in that space. In programming language with dynamic memory management, this issue can be solved by a garbage collector which automatically frees up memory that is no longer in use. Different garbage collectors have different strategies for managing memory, but they all aim to reduce fragmentation and improve the performance of the program. In this thesis, we will explore the use of free-list allocators in garbage collectors, and its integration in one of Java's latest garbage collectors, ZGC. ZGC uses bump pointers to allocate memory which allows for fast allocations. However, it is known that bump pointers can cause fragmentation in the memory. The goal of using a free-list allocator is to let the garbage collector know that there is free space available in between allocations, which is something that bump pointers are not able to do. We will investigate the feasibility of using such an allocator and how it can be used to compact memory.