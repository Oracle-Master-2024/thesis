
%%% Local Variables:
%%% mode: latex
%%% TeX-master: "main"
%%% End:
Memory fragmentation is an issue in programs using dynamic memory allocation. Java uses garbage collectors (GCs) to mitigate this problem, with ZGC being one of the latest additions. ZGC addresses fragmentation by relocating objects in memory to create densely packed allocations. However, ZGC's bump-pointer allocation method, while efficient, leaves certain free memory areas unused. This thesis proposes a solution for compacting memory in ZGC using free-list based allocators, allowing allocations to utilize previously unreachable memory. The trade-off involves a more computationally intensive allocation method but results in more available memory for compaction.

Two free-list based allocators, an optimized TLSF (Two-Level Segregated Fit) allocator and a Buddy allocator, were integrated into ZGC. These allocators were evaluated through benchmarking, focusing on allocation throughput, fragmentation, and overall performance. The results indicated that the TLSF allocator showed promise in efficiently utilizing fragmented memory, demonstrating low fragmentation and maintaining performance levels comparable to the bump-pointer method.

The findings suggest that integrating free-list based allocators into ZGC can potentially improve memory utilization without significantly impacting performance. This approach opens new avenues for enhancing garbage collection techniques and optimizing memory management in the JVM. Future work will explore further optimizations and the application of these allocators in other phases of garbage collection.
% in programs that run for longer periods of time. By continuously alternating between allocating and releasing memory, the program can cause the memory to become fragmented, meaning objects are no longer stored in a contiguous block of memory. The free regions of memory between objects can then be hard to utilize, because certain objects might not fit in that space. In programming language with dynamic memory management, this issue can be solved by a garbage collector which automatically frees up memory that is no longer in use. Different garbage collectors have different strategies for managing memory, but they all aim to reduce fragmentation and improve the performance of the program. In this thesis, we will explore the use of free-list allocators in garbage collectors, and its integration in one of Java's latest garbage collectors, ZGC. ZGC uses bump pointers to allocate memory which allows for fast allocations. However, it is known that bump pointers can cause fragmentation in the memory. The goal of using a free-list allocator is to let the garbage collector know that there is free space available in between allocations, which is something that bump pointers are not able to do. We will investigate the feasibility of using such an allocator and how it can be used to compact memory.

