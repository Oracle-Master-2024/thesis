
%%% Local Variables:
%%% mode: latex
%%% TeX-master: "main"
%%% End:
%The goal of this thesis project is to investigate the possibility of using free-list allocators in garbage collectors. The project will focus on integrating a free-list allocator in the ZGC, a garbage collector available in the most recent release of the OpenJDK. Currently in ZGC, memory is allocated in regions using sequential allocators, discussed in more detail in Section~\ref{sec:background}. Although it is a very performance effective way of allocating objects, it is known that bump pointers can cause a lot of external fragmentation in the memory~\cite{TODO:bump}. By making use of a free-list allocator, it is possible to let the garbage collector know that there is free space available in between allocations, allowing for allocations inside of the externally fragmented memory. This will add some extra work load on the allocation of objects, but will allow for using memory more efficiently. The project will investigate the feasibility of using such an allocator, and whether it improves the throughput and or memory usage.
Garbage collection is a feature in many programming languages with dynamic memory allocations, where the task of a garbage collector is to keep track of which parts in the memory is currently being used by the running program. This makes it possible for the garbage collector to automatically free up unused memory allocations. Efficient memory management is useful for increasing the performance and resource usage of programs, since all of the memory utilization is heavily relying on how the memory is being kept track of by the underlying garbage collector.

Java is a programming language with dynamic memory management that makes use of garbage collectors in order to handle memory. Java is used by one of the largest services on Earth, Netflix. Operating within a runtime environment known as the Java Virtual Machine (JVM), Java allows users to configure the JVM for optimal performance, including the choice of garbage collector. In Mars 2024, a blogpost from Netflix was released~\cite{netflix:zgc}, announcing the benefits of switching from one garbage collector, to the latest, most modern garbage collector, generational ZGC. They found that the upgrade yielded an improvement of 6-8\% percent in CPU utilization, proving that new technologies can lead to great improvements.

This thesis will continue on the subject of further exploring possible improvements that can be made to the JVM and its garbage collectors. Specifically, the goal is to look at the possibility of implementing a new method for allocating objects inside fragmented memory where previously allocated objects have been found to be unreachable. In ZGC, this is not an available feature, since it opts for using bump pointers instead. In Section~\ref{sec:memory_allocation}, the difference between bump pointers and free-list based allocations is explained in more detail. With a new allocator in place which can handle allocations in free-lists, a new method for compacting memory would be available where objects can be allocated in non-contiguous regions of free memory, which is what this thesis will investigate the feasibility of doing.

The integration of a new allocation method meant that changes had to be done to how ZGC handles 