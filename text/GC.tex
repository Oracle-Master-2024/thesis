This section draws heavily from the book \textit{The Garbage Collection Handbook: The art of automatic memory management}~\cite{gchandbook}, by R. Jones et al., which provides a comprehensive overview of garbage collection methods, both historic and current.

As mentioned in Section~\ref{sec:memory_management}, garbage collection is a form of automatic memory management where the system identifies and cleans up unused objects, which are considered garbage. This removes the requirement of managing memory from the user, which reduces the possibility of memory-related issues occurring. However, this also removes some control from the user and could lead to suboptimal usage of system memory.

From the perspective of the GC, the user program can be perceived solely as a memory mutator. The GC's only concern is the management of memory access, rather than the program's functionality. As such, the threads of the user program are commonly referred to as mutator threads, or just mutators.

During GC execution, it may be necessary to temporarily halt the mutator threads to give the GC exclusive memory access. This is known as a stop-the-world approach. From the mutator's perspective, the GC appears to be atomic, although extended pauses may not be desirable. The GC could also operate concurrently and in parallel with the mutator threads. This leads to increased complexity and challenges in identifying live objects. Most GCs fall somewhere in between these ends, executing concurrently with the mutator threads while occasionally requiring brief pauses in their execution.

There exist numerous unique garbage collection strategies. One is the mark-sweep algorithm, which begins by identifying a set of accessible ``root'' objects. These objects are then marked as live and scanned to identify any new linking objects. This process continues recursively until all live objects have been identified. Afterward, the heap is swept, freeing all objects that were not previously marked.

A problem with the mark-sweep algorithm, however, is that it does not relocate objects, which can result in higher levels of external fragmentation. The mark-compact algorithm addresses this issue. Like mark-sweep, it traverses the heap and marks all live objects. However, instead of sweeping, live objects are relocated to either the top or bottom of the heap. This approach reduces fragmentation and can be managed using a simple bump-pointer allocator. It may be necessary to repeat this process multiple times depending on the positioning of live objects, as they could potentially collide with the moving operation.

This problem is solved by a copying garbage collector. The copy algorithm divides the heap into many logical units. Allocations are made in the same way as mark-compact, allocating onto units and marking them as not empty. When garbage collection is initiated, the object is copied to another empty unit rather than to the edge of the heap. Once all living objects have been copied from the unit, it is again marked empty. Heap size is significantly reduced when using the copying algorithms with few units. Using it with many units instead could lead to needing garbage collection to be done more often.

%%% Local Variables:
%%% mode: latex
%%% TeX-master: "main"
%%% End:
