Här beskriver ni i princip er problemformulering.  I detta avsnitt ska framgå syfte, mål, och motivation med projektet. 
Dessa behöver dock \emph{\textbf{inte}} vara separata underrubriker.

\paragraph{Syfte.} Vart strävar projektet? Vad är det övergripande målet, nyttan, effekterna av projektet?  (t.ex. bättre hälsa genom koll på kosthållning, enklare planering av studier\ldots)
\paragraph{Mål.} Vad ska konkret levereras/utföras av projektet, för att ta oss närmare syftet?
\paragraph{Motivation.}  Varför är just ert projekt viktigt?  Vilka är det viktigt för, vilka externa intressenter finns?  Hur stort är problemet, vad är följden av att det inte är löst, hur bra vore det att lösa?  Vilken ``lucka'' i området täcker ni?
Varför är er lösning bättre/annorlunda än andras?

Undvik att ta upp er interna motivation, som väldigt sällan är intressant för läsaren.

Se till att ni i detta avsnitt övertygar läsaren om att problemet finns, att det inte är löst, och att det är viktigt att lösa. Ju starkare argumentation och motivation (med källor) dess bättre.
\begin{itemize}
\item Visa att det finns ett problem.
\item Visa att problemet är viktigt att lösa, att det behöver lösas.
\item Visa att problemet inte redan är löst.
\end{itemize}

I det här avsnittet kan ni också börja beskriva etiska aspekter, hållbarhetsaspekter, och dataskyddsaspekter, men det finns förstås flera andra naturliga ställen att ta upp dem (\emph{till exempel} sektionerna~\ref{sec:metod} och \ref{sec:systemstruktur} (för att motivera olika val), \ref{sec:krav} (för att motivera olika krav), eller \ref{sec:resultat} och~\ref{sec:slutsatser} för att reflektera över resultatet, men kanske redan i sektion~\ref{sec:introduktion} eller \ref{sec:bakgrund}).

Det är helt OK (och bra!) att också beskriva negativa/kritiska aspekter av ert projekt och arbete, inte bara positiva/goda. 

\emph{Använd kursmaterialet} för att få stöd att utveckla dessa aspekter. Läs t.ex. om forskningsetik~\cite{vr:forskningsetik} och dataskyddsprinciper\footnote{Integritetsskyddsmyndighetens webbsajt: \url{https://www.imy.se}}.
Använd stödfrågor som t.ex följande. \textbf{Se inte detta som ett frågeformulär}, men se till att hantera frågeställningarna -- kanske inte just här, men på relevanta ställen i rapporten.
\begin{itemize}
\item Vilka är \emph{direkta} intressenter och hur påverkas de? (användare, företag, kunder)
\begin{itemize}
\item  Vad krävs för att kunna använda er lösning? (kunskap, förmågor, resurser)
\item  Vilka exkluderar ni?
\item  Vad underlättar ni och vad gör ni svårare?
\item  I vilka sammanhang kan er lösning användas och inte användas?
\end{itemize}
\item  Vilka är \emph{indirekta} intressenter och hur påverkas de? (familj, samhälle, konkurrenter)
\item  Kan tekniken användas för ``fel'' syften?
\item Om ni behandlar personuppgifter, varför behöver ni göra det, och för vilka ändamål? Har ni laglig grund, och följer ni GDPR-principerna?
\item Hur skyddar ni personuppgifter, hur minimerar ni mängden, och hur minimerar ni tiden de bevaras?
\item Vilka resurser behöver systemet för att fungera? VIlka behövs för att underhålla systemet? Vad krävs för att avveckla systemet?
\item  Hur ser ett samhälle ut där er lösning används i stor skala?
\end{itemize}

När ni beskriver hållbarhetsaspekter av ert projekt, utgå från själva projektet och hur det hjälper olika typer av hållbarhet, och haka sen in referenser till FNs hållbarhetsmål om det är relevant. Att börja i FN-änden och sen beskriva hur ni bidrar till olika hållbarhetsmål ger ofta ett mer påklistrat och artificiellt intryck.


%%% Local Variables:
%%% mode: latex
%%% TeX-master: "rapport-mall"
%%% End:
