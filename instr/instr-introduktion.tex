Beskriv åtminstone samma saker som i abstract, men mer utförligt -- typiskt 1-2 sidor. Spara tekniska detaljer till senare, eftersom läsaren inte är insatt än.

\begin{enumerate}
\item Vilket är området ni arbetar inom? Vad är problemet, ämnet, kontexten? 
\item Varför är problemet viktigt/intressant att lösa?
\item Hur angreps/löstes problemet? 
\item Vad är resultaten, hur väl löstes problemet?
\item Hur bra blev resultaten, hur användbara är de?
\end{enumerate}

Tänk på att börja introduktionen med en mening eller ännu hellre ett helt stycke som ``fångar'' läsaren och motiverar läsaren att fortsätta läsa.  \emph{Vi har valt att göra ett projekt om X} är relevant för er, men kommer inte att vilja få någon att läsa vidare.  Försök åtminstone få med någon slags bakgrund/kontext och (helst) motivation att fortsätta läsa.  Typ \emph{X är ett programspråk som tagit världen med storm.  Vi vill utforska om man kan kombinera X med Y för att göra\ldots}

Se till att ni \emph{kommer till kritan snabbt} – man vill inte läsa igenom två stycken text innan man får veta vad ni tänker göra i ert projekt.  Börja t.ex. \emph{inte} med att presentera alla idéer ni inte valt – läsaren vill veta vad ni ska göra, inte vad ni inte ska göra. 
Använd gärna en bild som visar vad det är ni åstadkommit.

Översiktlig beskrivning av systemet och dess features ska vara under systemdesign / systemstruktur, inte i introduktionen.

Introduktionen bör vara begriplig för t.ex.~en student i årskursen under, och gärna för en ännu bredare läsarkrets.

Avsluta gärna med en överblick över hela rapporten, där ni \emph{kortfattat} beskriver vad de olika kapitlen handlar om.

%%% Local Variables:
%%% mode: latex
%%% TeX-master: "rapport-mall"
%%% End:
