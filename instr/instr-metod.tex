Här beskriver ni vilka metoder/verktyg/tekniker/approacher ni använt för att lösa problemet / besvara frågeställningen.  Vilka metoder har ni konkret använt för att lösa problemet/bygga systemet?  Vilka tekniker/verktyg använde ni?

Observera att det inte är samma sak som att beskriva \emph{hur} ni använde teknikerna/verktygen: det kommer i Del X, implementationsdelen (se avsnitt~\ref{sec:del-x}).

Glöm inte att \emph{motivera} era val av metoder. Finns det flera rimliga alternativ? Beskriv varför ni inte valt dem (t.ex.~varför er valda metod är bättre).
Visa att det är rimligt att använda just detta tillvägagångssätt.
Det gäller även i det fall det är givet på förhand vilken teknik ni ska använda (t.ex. vilket programspråk) för att det ska passa i ett sammanhang eller existerande system (t.ex. ett som ny bygger vidare på). 
Även om man har en given teknik kan man alltså behöva förklara att en annan egentligen vore bättre -- men att omständigheter gör att ni ändå måste välja den givna tekniken.

Det är ofta bra att börja med att förklara vad ni valt för teknik/verktyg, och därefter motivation och alternativ. Om man börjar med alla alternativ och väntar med att förklara vad man valt till sist, blir det inte lika enkelt att läsa.

Detta avsnitt ska \emph{inte} innehålla information om hur gruppen organiserat arbetet (github, trello, jira\ldots) \emph{om} det inte är relevant för resultatet (och det är det oftast inte).

Använd tydliga underrubriker, t.ex. ``Ramverk för webbplatsen'' snarare än ``Wordpress'', eller ``Databashanterare'' snarare än ``MongoDB''.~\cite{mongo}.

%%% Local Variables:
%%% mode: latex
%%% TeX-master: "rapport-mall"
%%% End:
